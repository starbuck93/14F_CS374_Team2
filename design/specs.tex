\documentclass[11pt]{article}
\usepackage{hyperref}
\usepackage{graphicx}
\usepackage[margin=0.5in]{geometry}

\hypersetup{linkcolor=blue,colorlinks=true}

\begin{document}
\begin{center}
\LARGE{\textbf{Specifications Document}}\\
\normalsize{Team 2 - CS374 - Fall 2014}
\end{center}
\vspace{.1in}

In this document, we describe the specifications for our Schedule Conflict Calculator application.
\vspace{.2in}

\LARGE Table of Contents \\

\normalsize
\begin{tabular}{| l || r |}
  \hline
  \nameref{sec:reqs} & Describes requirements of our product \\ \hline
  \nameref{sec:scenes} & Describes user interaction with our product \\ \hline
  \nameref{sec:dataflow} & Describes the interaction of modules within our product \\ \hline
  \nameref{sec:glossary} & Glossary of terms \\
  \hline
\end{tabular}

\pagebreak[4]

\section{Requirements} \label{sec:reqs}

Our Schedule Conflict Calculator Web application will function as an extension to other academic administration technologies. Given
a set of inputs related to moving a course from one time block to another, the application determines the number of student
scheduling conflicts that would arise in some number of move scenarios. The application handles a number of different move
scenarios, allowing the user to decide which scenario to test.

\pagebreak[4]

\section{Use Cases and Scenarios} \label{sec:scenes}

The user is allowed multiple scenarios for calculating schedule conflicts. For example, sometimes the user has a single section-change
description in mind and would like to see scheduling conflicts. In other instances, a user might know that they need to move a section
to another time and/or room, yet would like to know which room out of a set of possible candidates has the least number of conflicts.
\vspace{.2in}

\begin{figure}[h]
  \centering
  \includegraphics[width=0.5\textwidth]{diagrams/scenarioA.png}
  \caption{Describes a simple scenario for finding conflicts given a single section-change description}
\end{figure}

In the simple case, we consider a single section-change description supplied with a desired input section. Output will consist of the
number of conflicts as well as a detailed look at individual conflicts. Other input parameters might be needed to provide a more 
customized operation tailored to user needs.

\begin{figure}[h]
  \centering
  \includegraphics[width=0.5\textwidth]{diagrams/scenarioB.png}
  \caption{Describes a more complex scenario for finding conflicts given multiple section-change descriptions}
\end{figure}

Other scenarios are needed for more complex operations. A user might wish to provide a set of section-change descriptions to apply to
the input course in order to find the best change time. This set of descriptions may encompass multiple rooms and buildings and also
take into consideration moving courses within the set of section-change descriptions.

\pagebreak[4]

\section{Data Flow} \label{sec:dataflow}

This section describes the interaction of various modules within our application. We map the flow of data through the different
modules of the application, demonstrating how it is used and transformed.

\begin{figure}[h]
  \centering
  \includegraphics[width=1\textwidth,scale=1.5]{diagrams/dataflow.png}
  \caption{Data flow diagram for main application operation}
\end{figure}

\pagebreak[4]

\section{Glossary} \label{sec:glossary}

\begin{description}
  \item[association] The set of courses whose location matches and time duration overlaps a section-change description
  \item[building] A collection of rooms
  \item[course] A general description of a class of sections
  \item[data-handler] An object that executes a scenario using data obtained from the database
  \item[floating-section] A section that may be potentially moved from either its current time, location or both
  \item[locked-section] A section that cannot be moved from its current time and location
  \item[professor] An instructor of a course; there is one professor per section
  \item[office-hours] A time block associated with a professor; a potential conflict item to consider
  \item[room] The meeting place of a section
  \item[scenario] A section change operation (e.g. moving a section from one time to another)
  \item[section] An instantiation of a course
  \item[section-change description] A tuple that specifies a desired change to a section's time and location.  Non-room changes have a hierarchy; change in a section's day implies change in time.  A change in professor implies change in day and change in time.
  \item[student] A participant in a section
\end{description}


\end{document}
